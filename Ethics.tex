\documentclass[10pt,twocolumn]{article} 

\usepackage{oxycomps} % use the main oxycomps style file

\bibliography{references}

\pdfinfo{
    /Title (Ethics; AI with Chinese chess)
    /Author (Haotian wang)
}

\title{Ethics: AI with Chinese chess}

\author{Haotian wang}
\affiliation{Occidental College}
\email{hwang2@oxy.edu}

\begin{document}

\maketitle

	In this paper, the main scope is to focus on the ethical issues and concerns regarding using training a machine learning AI to learn Chinese chess and compete with actual humans. The world is almost driven by technologies today, countless innovative products are being invented and introduced to people every day. When the project is designed with ethical and moral standards, it is essential to prevent unequal, unethical, injustice, and harmful events from happening. During the early stages of computer technology, ethical concerns about computers were virtually non-existent because computers were not nearly as complex as they are now. However, ethical circumstances about computers and cyber technology are unavoidable in contemporary civilization. Computer technology permeates every area of our daily life. Different computer technology offers distinct capabilities that enable people to carry out daily tasks effectively and quickly. In recent decades, computer scientists have been trying to produce multiple machine learning AI in chess games and other competitive sports in favor of testing the limitations of both humans and machines. We now live in a world dominated by capitalism and advanced technology. With the more advanced science and technologies, humans tend to apply these techs in various fields to save cost or provide convenience. However, while technologies such as AI is being implemented in different areas, different ethical consideration must be considered and create precautions for it. This paper will examine ethical issues, including power, accessibility, resource consumption, upkeep, and maintenance.

	The general ethical discussion going on for years is should AI exist? Scientists and the general public have discussed this controversial topic for decades. Simple AI and machine learning can provide people convenience in daily life, such as making a living easier using AI-controlled intelligent home devices. More complicated AI may be applied to auto-driving cars or search engines to provide more accurate search results. Scientists even pushed the limitation of AI into making decisions for humans. For instance, A corporate institution may use ai to help them select the employees they need. A college may use AI to filter out those students who aren't qualified for their standards. A court may use AI to judge human crimes based on laws. These examples demonstrate that with the advancement of AI technology to this day, it can replace humans in many positions while providing more efficiency to work. Machines, robots, and algorithms are slowly replacing some people's positions in modern society. The creators and users of AI must set ethical and moral standards for these technologies if the AI trend is unstoppable\cite{EthicalIssue}. It is clear that ethics are not only vital but also critical in reshaping these powerful algorithms in favor of humanity. Machines learn ethics through the programming that is programmed into them. As a result, the creators of these so-called AI algorithms should not take this lightly. Where we are today is the result of millions of years of evolution. When compared to the development of AI fields, we are not nearly a century old since the first computer was invented. The internet is only three decades old. So you can picture the rate of change in the world of artificial intelligence. Without ethics, a field as fast-paced as this would spell tragedy if it fell into the wrong hands\cite{BillGates}.

	There is a potential power ethic issue in the project of training an AI to learn and mimic the human mindset in playing chess. The power issue strands for providing some or a group of people unequal power over others. As a project part of the trend of AI and machine learning, this project potentially acts as a small part of the grand scheme. While the ability of AI can beat the best of humans in other board games, such as Alpha GO winning Lee Sedol in the 'GO' game during the 2016 march, it inspired others to do the same in other chess or board games\cite{AlphaGo}. Even real sports as everyone knows that chess board games and any other sports games are supposed to be fair to play. In the context of this paper, the Chinese chess game should be assumed to be honest and unbiased. With the pandemic happening, more and more people choose to play the Chinese chess game online and compete with other players. A small group of people uses AI to help them win higher-rank games, resulting in a series of unfair games. Evidentally, Youtube held dozens of video tutorials on using Online AI to cheat in online chess games. This potentially distributes power to those who use AI in online competitions, Which could be defined as cheating. Moreover, the player playing against those cheaters may have no idea about that, resulting in a bad mood and even doubting their abilities in playing the game. The charming part of a competitive game is about both winning and losing. If someone is only there to experience loss, it may cause them to quit the game forever. This type of cheating is unethical and unacceptable and wholly abandoned the common sense of maintaining a fair competition game environment. The purpose of creating this AI Chinese chess project is to help train players' chess skills while practicing it. Using this project to cheat in online competition games should not be tolerated, yet there are only a few ways to prevent this from happening. The solution found is limited but could potentially be helpful. Since this project will be hosted online website for Chinese chess players to practice, it became difficult for the team to monitor know-how uses may use it. Therefore, a website plugin is required to be installed to be able to use the AI Chinese chess website. This plugin will ask for users' consent before download and installation. The plugin will contain a simple URL collection and comparison algorithm. It will simply collect all the URLs or websites running on the user's browser and compare them with all the online Chinese chess game websites or apps. If the plugin detects potential cheating attempts, the plugin will provide feedback to the server and refuse to provide practice service for the current users. The URL and website data will only be stored locally and deleted once the plugin is quitted. 

	The project must include accessibility features for more minority or disabled communities in order to serve those who need to practice with AI. About 15 percent of the people in the world have some kind of disability, as reported by The world Bank, indicating there is a large group of people who need accessibility features in some format in daily life\cite{WorldBank}. The physical Chinese chess game isn't limited to the 'normal' group of people, and it also has different forms of existence, including voice format and braille alphabet engraved on the chess piece for disabled people. This AI-Chinese chess project will adapt the voice format for the vision-impaired community. However, because this project is done online and requires the InternetInternet to run, it is impossible to provide the damaged vision community service with the braille alphabet. Giving accessibility to those minority groups and providing them an opportunity to practice with AI is essential. It encourages them to use our AI, providing them with enjoyable experiences. The chess game has a voice format where the users can use a microphone to command the AI to move the chess piece and receive feedback from the speaker. In addition to that, the website will also provide a chess move reader for those who wish to listen to the chess moves. Avoiding relying solely on sound to deliver vital information is also essential to assist these individuals. Instead, offer additional media for support in parallel, such as subtitles with full captions for videos and transcripts for audio. Subtitles and transcripts should be complete and not leave out any important lines. The program will also provide an overall zoom in and out function and a sectional zoom in and out for older adults with worsened vision. Accessibility features aren't the main focus of this project. Yet, it is essential to provide everybody equal opportunities, and put it in simple words, that is the right thing to do. 
	
	While AI is learning how the human brain functions and uses it to either assist people in work, life, or provide entertainment, these AI's consume tons of energy resources to train the model and cause environmental concerns. For instance, on 45 terabytes of data, OpenAI trained its GPT-3 model. According to the US Energy Information Administration, the average household uses 10,649 kWh per year. As a result, training the final version of MegatronLM consumed nearly the same amount of energy as three homes use in a year\cite{EnergyConsumption}. The heavier the data set is used to train, the more energy and resources it consumes. As the trend of AI rises, a great deal of AI models has been developed and trained in recent decades. Even though these AI models may contribute to society, individuals, or organizations in certain ways, the creator must not ignore the potential environmental issues it creates. The increased use of resources may cause a more toxic environment on Earth and speed up climate change. Consume energy results in the production of CO2, the principal greenhouse gas released by humans. In the atmosphere, greenhouse gases such as CO2 trap heat at the Earth's surface, increasing the Earth's temperature and upsetting sensitive ecosystems. Protecting the environment should be everyone's responsibility. Under the condition that training artificial intelligence models contributes to the advancement of human society and technology, scientists must consider optimizing algorithms and data usage to reduce the training capacity and energy consumption. The AI-Chinese chess project tries to use the minimum data required to train the model and use renewable energy to reduce CO2 emissions. Just as McGovern said, "we must factor in the earth experience" while creating and training AI.  

	The upkeep and maintenance of the AI are also essential and must be done responsibly. Since the creation of AI and the training process uses a considerable amount of energy, then it must not be wasted and left unused. The goal of AI is to contribute the human society. Thus AI should be able to keep working as long as there are demands and maintained adequately. As with any piece of machinery, an AI model must be tuned and updated on a regular basis to maintain performance objectives. Failure to conduct this critical step may degrade model accuracy over time. Artificial intelligence models are trained using previous data. If these models were run in a static setting with static data, model performance would remain constant — indefinitely. However, models are rarely performed in static situations; instead, they are subjected to constantly changing surroundings and variables. These changes gradually degrade model performance over time, as the model lacks predictive power for interpreting unknown data. This performance degradation is referred to as model drift\cite{AIModelMaintenance}. Once the AI-Chinese chess project is trained, tested, and tuned before being uploaded onto the internet, monitoring the model will be done on weekly bases. The team will record the accuracy data and winning rate. Suppose the accuracy and winning rate drops irregularly or simply the users are getting better at the game. In that case, the team will adjust and optimize the model to keep it working as intended. 

	Overall, Ethical considerations exist in the technology field, specifically in the AI-Chinses chess project, several issues need to be considered and improved. The project consists of potential unequal power distribution issues through cheating in competitive online games. The solution to this potential cheating issue should be able to solve by an online URL detection plugin that requires users' consent. Furthermore, The website will install accessibility features, including voice commands and audio feedback, zoom in and out functions, and subtitles for those who need them. Excessive energy consumption is a serious issue that needs to be considered. Using energy to train AI models is not avoidable, but using fewer data for less train time along with renewable energy may be the solution to this issue. Lastly, Regular AI model monitoring activity must be done and further maintenance by updating and optimizing the model if necessary. 


\printbibliography 

\end{document}